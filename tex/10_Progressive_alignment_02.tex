%\documentclass[12pt]{article}
%\usepackage[a4paper, margin=1in]{geometry} 
%\usepackage{graphicx} 
%\usepackage{hyperref}
%\usepackage{float}
%\usepackage{multicol}
%\usepackage{multirow}
%\usepackage{amsmath}
%\usepackage{amssymb}
%\usepackage[ruled]{algorithm2e}
%\usepackage[font=small, labelfont=bf]{caption}
%
%\begin{document}

%
% Alignment clustering
%
\subsection{Alignment clustering}
Alignment clustering can be used even when accurate phylogenic trees are not available. 

%
% Clustering methods
%
\subsubsection*{Clustering methods}
\begin{itemize}
\item Linear clustering
\item Linkage clustering
\end{itemize}

%
% Linear clustering
%
\subsubsection*{Linear clustering}
\begin{enumerate}
\item Start with an alignment with a single sequence
\item Add a single sequence to the alignment
\item Repeat until no sequence is left
\end{enumerate}

%
% Selection of the next sequence
%
\subsubsection*{Selection of the next sequence}
\begin{itemize}
\item Most similar to the one already in the alignment
\item Most similar to the average sequence in the alignment
\end{itemize}

%
% Pseudo-code of linear progressive alignment (general progressive alignment)
%
\subsubsection*{Pseudo-code of linear progressive alignment (general progressive alignment)}

\begin{algorithm}[H]
  \SetKwInOut{U}{$\mathrm{U}$}
  \SetKwInOut{A}{$\mathcal{A}$}
  \SetKwData{dRAB}{$\mathrm{R_{a,b}}$}
  \SetKwData{dG}{$\mathrm{g}$}
  
  \BlankLine
    
  \U{Set of sequences not aligned}
  \A{Current alignment}
  
  \BlankLine \BlankLine
  
   $\mathrm{U} \leftarrow \{s_1, s_2, ... s_n\}$;\\
   Choose two sequences s and t from U;\\
   $\mathrm{U} \leftarrow \mathrm{U} - \{s, t\}$;\\
   $\mathcal{A} \leftarrow Align(s, t)$;

  \BlankLine \BlankLine

  \For{$i \leftarrow 1$ \KwTo $n-2$}{
      Choose a sequence s from U;\\
      $\mathrm{U} \leftarrow \mathrm{U} - \{s\}$;\\
      $\mathcal{A} \leftarrow Align(\mathcal{A}, s)$;
  }

  \SetAlgoRefName{\thesection.1}
  \caption{General progressive alignment}

\end{algorithm}

%
% Linkage methods
%
\subsubsection*{Linkage methods}
It requires the pair-wise alignment scores of all possible combinations. 

\begin{itemize}
\item Average linkage
\item Maximum linkage
\item Minimum linkage
\end{itemize}

%
% Example of linkage methods
%
\subsubsection*{Example of linkage methods}
It requires the pair-wise alignment scores of all possible combinations. \\

\noindent
Decide two alignments from the three alignments, $\mathcal{A}^1 = \{s^1\}, \mathcal{A}^2 = \{s^2\}$, and $\mathcal{A}^3 = \{s^3, s^4\}$, for clustering.\\

Pair-wise scores

\begin{table}[H]
\centering
\begin{tabular}{|l|l|l|l|l|}
\hline
   & s1 & s2 & s3 & s4 \\ \hline
s1 & 0  & 7  & 5  & 3  \\ \hline
s2 &    & 0  & 4  & 8  \\ \hline
s3 &    &    & 0  & 2  \\ \hline
s4 &    &    &    & 0  \\ \hline
\end{tabular}
\end{table}

%
% NEWPAGE
%
\newpage

Linkage selection

\begin{table}[H]
\centering
\begin{tabular}{lll}
Average linkage & $S(\mathcal{A}_1,\mathcal{A}_2) = 7$               & \checkmark  \\
                & $S(\mathcal{A}_1,\mathcal{A}_3) = (5+3)/2 = 4$       &   \\
                & $S(\mathcal{A}_2,\mathcal{A}_3) = (4+8)/2 = 6$       &   \\
                &                                                            &   \\
Maximum linkage & $S(\mathcal{A}_1, \mathcal{A}_2) = 7$               &   \\
                & $S(\mathcal{A}_1, \mathcal{A}_3) = \max(5, 3) = 5$ &   \\
                & $S(\mathcal{A}_2, \mathcal{A}_3) = \max(4, 8) = 8$   & \checkmark \\
                &                                                            &   \\
Minimum linkage & $S(\mathcal{A}_1, \mathcal{A}_2) = 7$               & \checkmark \\
                & $S(\mathcal{A}_1, \mathcal{A}_3 ) = \min(5, 3) = 3$ &   \\
                & $S(\mathcal{A}_2, \mathcal{A}_3) = \min(4, 8) = 4$   &  
\end{tabular}
\end{table}

%
% Exercise \thesection.1
%
\subsubsection*{Exercise \thesection.1}
Select two alignments from the three alignments: $\mathcal{A}^1 = \{s^1\}$, $\mathcal{A}^2 = \{s^2\}$, and $\mathcal{A}^3 = \{s^3, s^4\}$ for clustering.

\begin{table}[H]
\centering
\begin{tabular}{|l|l|l|l|l|}
\hline
   & s1 & s2 & s3 & s4 \\ \hline
s1 & 0  & 2  & 2  & 5  \\ \hline
s2 &    & 0  & 4  & 5  \\ \hline
s3 &    &    & 0  & 1  \\ \hline
s4 &    &    &    & 0  \\ \hline
\end{tabular}
\end{table}

\begin{enumerate}
\item Use the average linkage.
\bigskip 

\item Use the maximum linkage.
\bigskip 

\item Use the minimum linkage.
\end{enumerate}

\bigskip 

%\end{document}
